\documentclass[14pt]{article}
\usepackage{listings}
\usepackage{amssymb,amsmath,amscd,graphicx,latexsym,amsthm,hyperref,float,xypic, mathtools, graphicx}
\usepackage{titlesec,caption}
\usepackage{etoolbox}
\usepackage[utf8]{inputenc}
\usepackage[margin=1.1in]{geometry}

\usepackage{lmodern}
\usepackage[T1]{fontenc}
\usepackage{textcomp}

\usepackage{xcolor}
\usepackage{varwidth}
\setlength{\footskip}{.2 in}
\setlength{\headheight}{.1 in}
\setlength{\parindent}{0pt}

\hypersetup{
  colorlinks,
  citecolor=red,
  filecolor=blue,
  linkcolor=red,
  urlcolor=blue
}

\begin{document}

\title{Input data requirements for SPT \input{../spatialprofilingtoolbox/version.txt}}
\author{}
\maketitle

\begin{center}
\begin{varwidth}{0.4\textwidth}
\begin{enumerate}
  \itemsep0em
  \item{\label{ch}Channel information file}
  \item{\label{cells}HALO-exported cell table files}
  \item{\label{pheno}Phenotype definition table file}
  \item{\label{outcomes}Outcomes file}
  \item{\label{files}Files manifest file}
\end{enumerate}
\end{varwidth}
\end{center}

\newpage

\subsection*{\ref{ch} Channel information file}
A CSV file (\href{https://github.com/nadeemlab/SPT/blob/main/tests/data/elementary_phenotypes.csv}{example})
 with at least the following fields:

\begin{itemize}
  \itemsep0em
  \item[]{\colorbox{yellow!25}{\texttt{Name}}}
  \item[]{\colorbox{yellow!25}{\texttt{Column header fragment prefix}}}
\end{itemize}

\subsection*{\ref{cells} HALO-exported cell table files} A set of HALO-exported CSV files (\href{https://github.com/nadeemlab/SPT/blob/main/tests/data/2779f21192cb0ce1479b2bf7fb20ebba.csv}{example}), each with at least the following fields:

\begin{itemize}
  \itemsep0em
  \item[]{\colorbox{yellow!25}{\texttt{Image Location}}}
  \item[]{\colorbox{yellow!25}{\texttt{XMin}}}
  \item[]{\colorbox{yellow!25}{\texttt{XMax}}}
  \item[]{\colorbox{yellow!25}{\texttt{YMin}}}
  \item[]{\colorbox{yellow!25}{\texttt{YMax}}}
  \item[]{\colorbox{yellow!25}{\texttt{Cell Area}}}
  \item[]{\colorbox{yellow!25}{\texttt{Classifier Label}}}
\end{itemize}

In addition, for each channel, a column with the following form (\colorbox{yellow!25}{\texttt{$\langle$prefix$\rangle$}} refers to (\ref{ch})):

\begin{itemize}
  \itemsep0em
  \item[]{\colorbox{yellow!25}{\texttt{$\langle$prefix$\rangle$ Positive}}}
\end{itemize}


In addition, if intensity information is provided, SPT will make use of it in some situations when requested. In this case each CSV file should contain the following fields for each channel:

\begin{itemize}
  \itemsep0em
  \item[]{\colorbox{yellow!25}{\texttt{$\langle$prefix$\rangle$ Intensity}}}
\end{itemize}

\subsection*{\ref{pheno} Phenotype definition table file} A CSV file (\href{https://github.com/nadeemlab/SPT/blob/main/tests/data/complex_phenotypes.csv}{example}), with at least the following fields:

\begin{itemize}
  \itemsep0em
  \item[]{\colorbox{yellow!25}{\texttt{Name}}}
  \item[]{\colorbox{yellow!25}{\texttt{Positive markers}}}
  \item[]{\colorbox{yellow!25}{\texttt{Negative markers}}}
\end{itemize}

The positive and negative marker values should be semi-colon delimited strings, with each entry one of the \colorbox{yellow!25}{\texttt{Name}} values from part (\ref{ch}).

\subsection*{\ref{outcomes} Outcomes file} A TSV file (\href{https://github.com/nadeemlab/SPT/blob/main/tests/data/diagnosis.tsv}{example}), containing at least 2 columns:

\begin{itemize}
  \itemsep0em
  \item[]{\colorbox{yellow!25}{\texttt{Sample ID}}}
  \item[]{\colorbox{yellow!25}{\emph{anything}}}
\end{itemize}

The \colorbox{yellow!25}{\texttt{Sample ID}} values should correspond to one of the \colorbox{yellow!25}{\texttt{Sample ID}} values in part (\ref{files}).

\subsection*{\ref{files} Files manifest file} A TSV file listing all of the above files (\href{https://github.com/nadeemlab/SPT/blob/main/tests/data/file_manifest.tsv}{example}), containing at least the following fields:

\begin{itemize}
  \itemsep0em
  \item[]{\colorbox{yellow!25}{\texttt{File ID}} \hspace{0.5pc} An identifier for the file, unique within this dataset. Certain values are required: \colorbox{gray!20}{\texttt{Elementary phenotypes file}}, \colorbox{gray!20}{\texttt{Complex phenotypes file}}, for the corresponding files (\ref{ch}) and (\ref{pheno}).}
  \item[]{\colorbox{yellow!25}{\texttt{Project ID}} \hspace{0.5pc} A tag for this dataset or project, for downstream disambiguation.}
  \item[]{\colorbox{yellow!25}{\texttt{File name}} \hspace{0.5pc} The actual name to use to find this file in the dataset (without path information).}
  \item[]{\colorbox{yellow!25}{\texttt{Sample ID}} \hspace{0.5pc} The identifier for the sample associated with this data file, in case the data file is so associated (e.g. one of the HALO-exported cell CSVs).}
  \item[]{\colorbox{yellow!25}{\texttt{Data type}} \hspace{0.5pc} A tag for this file data type. The only required values are \colorbox{gray!20}{\texttt{HALO software cell manifest}} and \colorbox{gray!20}{\texttt{Outcome}}, for (\ref{cells}) and (\ref{outcomes}) respectively.}
\end{itemize}


\end{document}

